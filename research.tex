\documentclass[11pt,a4paper]{article}

\begin{document}
\title{Evaluation of different JavaScript characteristics}
\author{Omar Adel Brikaa \thanks{20206043 - S5 - brikaaomar@gmail.com}}
\date{}
\maketitle

\tableofcontents

\section{Introduction}
Javascript is a dynamically-typed, multi-purpose, multi-paradigm programming language. It has been mainly used to run
client-side code on web browsers. It has also grown into other areas of programming such as web server back-end
programming\footnote{https://nodejs.org/en/}. Below, I evaluate some different characteristics of JavaScript including
expressivity, type-checking, orthogonality, simplicity, syntax design and exception handling.

\section{Expressivity}
Expressivity is the ability to perform multiple computations using compact syntax. Following are some examples
of expressivity in Javascript:

\subsection{Compound assignment operators}
Like many other programming languages, JavaScript supports assigning a value to a variable based on its previous value
without repeating the variable name. The following are equivalent pieces of code:
\begin{verbatim}
    a = a + 1;
    a += 1;

    b = b * 1;
    b *= 1

    c = c + " world"
    c += " world"
\end{verbatim}

\subsection{Postfix and prefix increment, decrement}
JavaScript also supports incrementing and decrementing variables by one without the usage of compound assignment
operators:
\begin{verbatim}
    a += 1;
    a++;

    b -= 1;
    b--;

    let d = c; c += 1;
    let d = c++;

    f += 1; let e = f;
    let e = ++f;
\end{verbatim}

\subsection{Destructuring assignment}
If we want to assign some values in an array to variables, we can do so with a shorthand syntax:
\begin{verbatim}
    let arr = [1, 2, 3];
    let [a, b, c] = arr; // instead of:
    a = arr[0];
    b = arr[1];
    c = arr[2];
\end{verbatim}
Note that the variables must be ordered in the same way the values are in the array.
The same can be done with objects\footnote{Objects are discussed in detail in other sections}:
\begin{verbatim}
    let obj = {
        a: "Hello",
        b: "World"
    };
    let {a, b} = obj; // instead of
    a = obj.a;
    b = obj.b;
\end{verbatim}
Note that the variables must have the same name of the objects member names. Destructuring assignment can be used in more
powerful ways that are discussed in other sections.

\subsection{Spread syntax}
If we want to spread an array into another array, we can do so expressively with the spread syntax:
\begin{verbatim}
    let a = [2,3,4];
    console.log([1, ...a, 5, 6]);
    // expected output: [1, 2, 3, 4, 5, 6]
\end{verbatim}
More powerful usages of the spread syntax are discussed in other sections.

\subsection{Arrow functions}
JavaScript uses the `function' keyword to define a function; however, a more concise syntax has been introduced in
ES6\footnote{https://en.wikipedia.org/wiki/ECMAScript}. The new arrow function syntax is not only syntactically
different from the traditional function syntax; there are other differences which are discussed in other sections:
\begin{verbatim}
    // Traditional syntax
    function power(a) {
        return a * a;
    }

    // Arrow function
    const power1 = (a) => {
        return a * a;
    }

    // The curly braces can be omitted
    // Since the function directly returns a value
    const power2 = (a) => a * a;

    // The parentheses around the parameter
    // can also be omitted since it is only
    // one parameter
    const power3 = a => a * a;
\end{verbatim}

\subsection{for..of, for..in}
JavaScript provides expressive control statements that helps in iterating over an array:
\begin{verbatim}
    for (const i of [1, 2, 3, 4]) {
        console.log(i);
    }
    // instead of
    const arr = [1, 2, 3, 4];
    for (let i = 0; i < arr.length; ++i) {
        console.log(arr[i]);
    }
\end{verbatim}
It also provides compact syntax for iterating over an Object's keys
\begin{verbatim}
    let person = {
        name: "foo",
        age: 13,
    }
    for (const key in person) {
        console.log(person[key]);
    }
\end{verbatim}

\subsection{Functions inspired by functional programming}
JavaScript has multiple functions that are inspired by expressive functions in functional programming languages like
Haskell. For example, the map function takes an array and a
% TODO: Where?
unary function\footnote{Functions can be passed as parameters to other functions, this will be discussed in } and returns
an array that is the result of applying the function on each element in the given array:
\begin{verbatim}
    // Note the use of the arrow function syntax
    // And the definition of the map function on the fly
    let arr = [1, 2, 3].map(x => x * 2);
    // instead of
    let arr = [];
    for (const i of [1, 2, 3]) {
        arr.push(i * 2);
    }
\end{verbatim}
JavaScript also has a filter function that, given an array and a unary function,
returns a new array containing the elements on which the unary function returns true:
\begin{verbatim}
    const even = [1, 2, 3, 4, 5].filter(x => x % 2 == 0);
\end{verbatim}


\end{document}
