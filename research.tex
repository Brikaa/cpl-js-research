\documentclass[11pt,a4paper]{article}

\begin{document}
\title{Evaluation of different JavaScript characteristics}
\author{Omar Adel Brikaa \thanks{20206043 - S5 - brikaaomar@gmail.com}}
\date{}
\maketitle

\tableofcontents

\section{Introduction}
Javascript is a dynamically-typed, multi-purpose, multi-paradigm programming language. It has been mainly used to run
client-side code on web browsers. It has also grown into other areas of programming such as web server back-end
programming\footnote{https://nodejs.org/en/}. Below, I evaluate some different characteristics of JavaScript including
expressivity, orthogonality, simplicity, syntax design and exception handling.

\section{Expressivity}
Expressivity is the ability to perform multiple computations using compact syntax. Following are some examples
of expressivity in Javascript:

\subsection{Compound assignment operators}
Like many other programming languages, JavaScript supports assigning a value to a variable based on its previous value
without repeating the variable name. The following are equivalent pieces of code:
\begin{verbatim}
    a = a + 1;
    a += 1;

    b = b * 1;
    b *= 1

    c = c + " world"
    c += " world"
\end{verbatim}

\subsection{Postfix and prefix increment, decrement}
JavaScript also supports incrementing and decrementing variables by one without the usage of compound assignment
operators:
\begin{verbatim}
    a += 1;
    a++;

    b -= 1;
    b--;

    let d = c; c += 1;
    let d = c++;

    f += 1; let e = f;
    let e = ++f;
\end{verbatim}

\subsection{Destructuring assignment}
If we want to assign some values in an array to variables, we can do so with a shorthand syntax:
\begin{verbatim}
    let arr = [1, 2, 3]

    let [a, b, c] = arr; // instead of:
    a = arr[0];
    b = arr[1];
    c = arr[2];
\end{verbatim}
Note that the variables must be ordered in the same way the values are in the array.
The same can be done with objects\footnote{Objects are discussed in other sections}:
\begin{verbatim}
    let obj = {
        a: 'Hello',
        b: 'World'
    }
    let {a, b} = obj; // instead of
    a = obj.a
    b = obj.b
\end{verbatim}
Note that the variables must have the same name of the objects member names. Destructuring assignment can be used in more
powerful ways that are discussed in other sections.

\subsection{Spread syntax}



\end{document}
