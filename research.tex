\documentclass[11pt,a4paper]{article}
\PassOptionsToPackage{hyphens}{url}\usepackage{hyperref}

\begin{document}
\title{Evaluation of different JavaScript language design characteristics}
\author{Omar Adel Brikaa - 20206043\thanks{Omar Adel Abdel Hamid Ahmed Brikaa - S5 - brikaaomar@gmail.com}}
\date{}
\maketitle

\tableofcontents

\section{Introduction}
JavaScript is a dynamically-typed, multi-purpose, multi-paradigm programming language.
It has been mainly used to run client-side code on web browsers.
It has also grown into other areas of programming such as web server back-end programming\footnote{https://nodejs.org/en/}.
This article, evaluates some different characteristics of JavaScript language design including:
types, expressivity, orthogonality, simplicity, syntax design and exception handling.

The code examples in the article were tested in a chromium-based\\browser and might not work in NodeJS and other back-end
runtimes due to the lack of the global `window' object.
The article discusses some features which were introduced in ES6\footnote{https://en.wikipedia.org/wiki/ECMAScript}.

\section{Types}

\subsection{\label{dynamic_typing}Dynamic typing}
The JavaScript interpreter assigns a variable a type at runtime based on its value.
\begin{verbatim}
    let a = "foo";
    console.log(typeof a); // "string"
    a = true;
    console.log(typeof a); // "boolean"
    a = 12;
    console.log(typeof a); // "number"
    a = { hello: "world" };
    console.log(typeof a); // "object"
\end{verbatim}
When defining a function, the parameters types are determined at runtime at each call:
\begin{verbatim}
    function add(a, b) {
        return a + b;
    }
    console.log(add(1, 2)); // 3
    console.log(add("hello ", "world")); // hello world
\end{verbatim}
Arrays can have elements of different types:
\begin{verbatim}
    const arr = [1, "foo", true, { hello: "world" }];
    arr.forEach((x) => console.log(typeof x));
    // Expected output:
    // number
    // string
    // boolean
    // object
\end{verbatim}

\subsection{Weak typing}
When the types of variables mismatch, the interpreter tries to convert the type of one variable into the other.
This is known as type coercion.
\begin{verbatim}
    const a = "1";
    const b = 2;
    console.log(a + b);
    // "12", b was coerced into a string

    const arr = [1,2,3];
    console.log(arr.slice("1"));
    // [2,3], "1" was coerced into a number
\end{verbatim}

\subsection{Type checking}
When a type can not be coerced into another type, JavaScript throws a TypeError at runtime.
\begin{verbatim}
    const a = null;
    a.sayHello(); // TypeError: cannot read properties of null

    const b = 12;
    b(); // TypeError: b is not a function
\end{verbatim}

\section{Expressivity}
Expressivity is the ability to perform multiple computations using compact syntax.
Following are some examples of expressivity in JavaScript:

\subsection{Compound assignment operators}
Like many other programming languages, JavaScript supports assigning a value to a variable based on its previous value
without repeating the variable name. The following are equivalent pieces of code:
\begin{verbatim}
    a = a + 1;
    a += 1;

    b = b * 1;
    b *= 1

    c = c + " world"
    c += " world"
\end{verbatim}

\subsection{Postfix and prefix increment, decrement}
JavaScript also supports incrementing and decrementing variables by one without the usage of compound assignment
operators:
\begin{verbatim}
    a += 1;
    a++;

    b -= 1;
    b--;

    let d = c; c += 1;
    let d = c++;

    f += 1; let e = f;
    let e = ++f;
\end{verbatim}

\subsection{Destructuring assignment}
If we want to assign some values in an array to variables, we can do so with a shorthand syntax:
\begin{verbatim}
    let arr = [1, 2, 3];
    let [a, b, c] = arr; // instead of:
    a = arr[0];
    b = arr[1];
    c = arr[2];
\end{verbatim}
Note that the variables must be ordered in the same way the values are in the array.
The same can be done with objects\footnote{Objects are discussed in detail in other sections}:
\begin{verbatim}
    let obj = {
        a: "Hello",
        b: "World"
    };
    let {a, b} = obj; // instead of
    a = obj.a;
    b = obj.b;
\end{verbatim}
Note that the variables must have the same name of the objects member names. Destructuring assignment can be used in more
powerful ways that are discussed in other sections.

\subsection{\label{spread_syntax}Spread syntax}
If we want to spread an array into another array, we can do so expressively with the spread syntax:
\begin{verbatim}
    let a = [2,3,4];
    console.log([1, ...a, 5, 6]);
    // expected output: [1, 2, 3, 4, 5, 6]
\end{verbatim}
More powerful usages of the spread syntax are discussed in other sections.

\subsection{\label{arrow_functions}Arrow functions}
JavaScript uses the `function' keyword to define a function; however, a more concise syntax has been introduced in
ES6. The new arrow function syntax is not only syntactically
different from the traditional function syntax; there are other differences which are discussed in other sections:
\begin{verbatim}
    // Traditional syntax
    function square(a) {
        return a * a;
    }

    // Arrow function
    const square1 = (a) => {
        return a * a;
    }

    // The curly braces can be omitted
    // Since the function directly returns a value
    const square2 = (a) => a * a;

    // The parentheses around the parameter
    // can also be omitted since it is only
    // one parameter
    const square3 = a => a * a;
\end{verbatim}

\subsection{\label{for_of_for_in}for..of, for..in}
JavaScript provides expressive control statements that helps in iterating over an array:
\begin{verbatim}
    for (const i of [1, 2, 3, 4]) {
        console.log(i);
    }
    // instead of
    const arr = [1, 2, 3, 4];
    for (let i = 0; i < arr.length; ++i) {
        console.log(arr[i]);
    }
\end{verbatim}
It also provides compact syntax for iterating over an Object's keys
\begin{verbatim}
    let person = {
        name: "foo",
        age: 13,
    }
    for (const key in person) {
        console.log(person[key]);
    }
\end{verbatim}

\subsection{\label{fp_functions}Functions inspired by functional programming}
JavaScript has multiple functions that are inspired by expressive functions in functional programming languages like
Haskell.
For example, the map function takes an array and a unary
function\footnote{Functions can be passed as parameters to other functions, this is discussed in another section}
and returns an array that is the result of applying the function on each element in the given array:
\begin{verbatim}
    // Note the use of the arrow function syntax
    // And the definition of the map function on the fly
    let arr = [1, 2, 3].map(x => x * 2);
    // instead of
    let arr = [];
    for (const i of [1, 2, 3]) {
        arr.push(i * 2);
    }
\end{verbatim}
JavaScript also has a filter function that, given an array and a unary function,
returns a new array containing the elements on which the unary function returns true:
\begin{verbatim}
    const even = [1, 2, 3, 4, 5].filter(x => x % 2 === 0);
    console.log(even);
    // Expected output: [2, 4]
\end{verbatim}

\section{Orthogonality}
Orthogonality is the ability to combine different constructs in a meaningful way without the need for special handling.
The following sections discuss some examples of orthogonality and lack of thereof in JavaScript:

\subsection{\label{objects}Objects}
JavaScript contains seven primitive data types: string, number, bigint, boolean, symbol, undefined and null.
Every non-primitive data type is an object,
and every primitive data type (except for null and undefined) has an object wrapper.
This allows for consistency while dealing with different data types and data structures.

An object is a collection of properties, and a property is an association between a name (or key) and a value.
A property's value can be a
function\footnote{\raggedright \url{https://developer.mozilla.org/en-US/docs/Web/JavaScript/Guide/Working_with_Objects}}.
The following is an example of an object:
\begin{verbatim}
    const obj = {
        name: "foo",
        sayName() {
            console.log("My name is " + this.name)
        }
    };

    obj.sayName(); // My name is foo
\end{verbatim}
Since every non-primitive type is an object, it is worth noting that
a function is an object with the additional capability of being callable,
and an array is an object with iterable integer-keyed properties that determine its length.

\subsection{\label{functions_first_class}Functions as first-class objects}
Functions can be treated as any other object.
They can be assigned to variables and constants, passed to other functions as parameters
and returned from other functions.
In fact, the arrow function example in \ref{arrow_functions}
assigns an anonymous function (a function without a name) to a constant (square1, square2, square3).

This is a functional programming principle known as `functions as first-class citizens'.
Section \ref{fp_functions} briefly hints to this principle.
The following is an example that shows features that have been discussed in
\ref{arrow_functions}, \ref{dynamic_typing}, \ref{for_of_for_in}, \ref{functions_first_class}, \ref{objects}:
\begin{verbatim}
    // Assigning an anonymous function to a constant
    const add = (a, b) => a + b;
    const subtract = (a, b) => a - b;
    const multiply = (a, b) => a * b;

    // A function that returns the appropriate operation
    // function according to the opString
    const chooseOperation = (opString) => {
        // Using an object to store the functions
        // keyed by their opString
        const ops = {
            "+": add,
            "-": subtract,
            "*": multiply
        };
        return ops[opString];
    }

    // A function that takes an operation
    // and its two arguments as arguments
    // and applies the operation to these arguments
    const applyOperation = (op, a, b) => op(a, b);

    const expr1 = "1 + 2";
    const expr2 = "2 * 3";
    const expr3 = "3 - 2";

    // For..of loop
    for (const expr of [expr1, expr2, expr3]) {
        // Destructuring assignment
        let [a, op, b] = expr.split(" ");
        // Dynamic typing, can change the value type
        // at runtime
        a = parseInt(a);
        b = parseInt(b);
        op = chooseOperation(op);
        console.log(applyOperation(op, a, b));
    }

    // Expected output:
    // 3
    // 6
    // 1
\end{verbatim}

\subsection{Spread syntax in different contexts}
The spread syntax discussed in \ref{spread_syntax} can be used consistently in different contexts.
It can be used to merge objects:
\begin{verbatim}
    const a = { fruit: "apple" };
    const b = { vegetable: "potato" }
    const c = { ...a, ...b };
    console.log(c);
    // { fruit: "apple", vegetable: "potato" }
\end{verbatim}
Since an array is an object, the spread syntax can be used to merge an array:
\begin{verbatim}
    const a = [1, 2, 3];
    const b = [4, 5, 6];
    const c = [...a, ...b];
    console.log(c);
    // [1, 2, 3, 4, 5, 6]
\end{verbatim}
It can be used to make a function accept an infinite number of parameters:
\begin{verbatim}
    // reduce is another functional programming
    // inspired function
    const multiply = (...xs) =>
        xs.reduce((acc, x) => acc * x, 1);

    console.log(multiply(2, 3)); // 6
    console.log(multiply(2, 3, 4)); // 24
\end{verbatim}

\subsection{Destructuring assignment in different contexts}
Likewise, the destructuring assignment can be used consistently in different contexts.
It can be in iterators like for..of:
\begin{verbatim}
    const objects = [
        { fruit: "apple", vegetable: "potato" },
        { fruit: "banana", vegetable: "pea" },
        { fruit: "pineapple", vegetable: "okra" }
    ];

    // Destructuring fruit and vegetable
    // from each object
    for ({ fruit, vegetable } of objects) {
        console.log(
            "The fruit is "
            + fruit +
            " and the vegetable is " + vegetable
        );
    }
    // Expected output:
    // The fruit is apple and the vegetable is potato
    // The fruit is banana and the vegetable is pea
    // The fruit is pineapple and the vegetable is okra
\end{verbatim}
It can be used in function parameters:
\begin{verbatim}
    const sayFoodInfo = ({ fruit, vegetable }) => {
        console.log(
            "The fruit is "
            + fruit
            + " and the vegetable is "
            + vegetable
        );
    }

    sayFoodInfo({ fruit: "date", vegetable: "tomato" })
    // Expected output:
    // The fruit is date and the vegetable is tomato
\end{verbatim}

\subsection{Destructuring assignment in tandem with spread syntax}
The destructuring assignment can be used in tandem with the spread syntax to
destructure the first n items of an array and put the remaining items in a second array
where n is the number of variables not preceded by `...' in the destructuring assignment.

Consider the following function that gets the last element in an array without using the length property nor indexing:
\begin{verbatim}
    // The parameters destructure the array
    // into the first element
    // and a second array containing the rest
    // of the elements
    const last = ([x, ...xs]) => {
        // Nothing could be destructured
        // (empty array)
        if (x === undefined)
            return [];
        // Nothing could be destructured from xs
        // Only one element, return it
        const [y] = xs;
        if (y === undefined)
            return x;
        // Still no base cases
        // last is the last of the remaining elements
        return last(xs);
    }

    console.log(last([1, 2, 3, 4, 5, 6])); // 6
\end{verbatim}

\subsection{typeof null (lack of orthogonality)}
The `typeof' JavaScript operator returns the type of a value in a string.
If the value is a primitive data type (\ref{objects}), it returns the name of that data type.
If the value is a non-primitive data type, it either returns `object' or
`function'\footnote{It does that although a function is an object, but that is another matter I will not discuss.}

One special case is `typeof null' or `typeof someNullishVariable'; it returns `object'.
This is a bug in the language's implementations that was not fixed due to the need for backward compatibility.
`typeof null' should be null since null is a primitive data type.

This leads to a lack of orthogonality when checking for nullish variables:
\begin{verbatim}
    for (const a of ["Hello", undefined, null, 123]) {
        if (typeof a === "string")
            console.log("a is a string");
        else if (typeof a === "undefined")
            console.log("a is undefined");
        // Notice how we need to handle nulls
        // differently here
        else if (a === null)
            console.log("a is null");
        else
            console.log("a is another data type");
    }
    // Expected output:
    // a is a string
    // a is undefined
    // a is null
    // a is another data type
\end{verbatim}

\section{Simplicity}
A programming language is simple if it consists of a small set of constructs to be learnt.
Because JavaScript is a fast-growing language with focus on backward compatibility,
many sacrifices to simplicity had to be made.
There exists many methods of doing the same thing in JavaScript.
A JavaScript programmer is expected to know many of them in order to be able to read JavaScript codebases.
The following are some examples of lack of simplicity in JavaScript:

\subsection{`this' keyword}
`this' keyword in JavaScript is used to refer to the current object but it behaves differently depending on where
it is used as shown in the following examples:

\subsubsection{`this' keyword in traditional functions}
The example in section \ref{objects} used `this' keyword to refer to the object in the `sayName' method.
The following modification to the example, however, will produce a wrong result:
\begin{verbatim}
    const obj = {
        name: "foo",
        sayName() {
            console.log("My name is " + this.name)
        }
    };

    const sayName = obj.sayName;
    sayName(); // My name is [different output]
\end{verbatim}

The last two lines in the code snippet were modified to assign the `obj.sayName' function to a constant,
and call the function using this constant.
The correct name was not outputted since `this' did not get bounded to the object.

When using traditional functions (not arrow functions) inside the object and calling them,
`this' gets bound to the object before the dot in the call.
In the above case, there was no object before the dot; hence it was bound to the global window object.
In fact, if the global window object contains a name property, it will be used:
\begin{verbatim}
    const obj = {
        name: "foo",
        sayName() {
            console.log("My name is " + this.name);
        }
    };

    window.name = "fake";

    const sayName = obj.sayName
    sayName(); // My name is fake
\end{verbatim}

\subsubsection{`this' keyword in arrow functions}
When using arrow functions inside an object and calling them, `this' is not bound to the function,
but rather gets its value from the outer scope:

\begin{verbatim}
    const obj = {
        name: "foo",
        sayName: () => {
            console.log("My name is " + this.name);
        }
    };

    window.name = "fake";
    obj.sayName(); // My name is fake
\end{verbatim}
In the above example, no value got bound to `this' in `sayName'; so the global `window' object was used.

The following example uses a traditional function in `sayName' that
returns an arrow function (\ref{functions_first_class}) which uses `this'.
\begin{verbatim}
    const obj = {
        name: "foo",
        sayName() {
            return () => {
                console.log("My name is " + this.name);
            }
        }
    };

    window.name = "fake";
    (obj.sayName())(); // My name is foo
\end{verbatim}
It produces the correct output since `this' got bound to `obj' in the traditional function,
and took the value of `obj' in the returned arrow function.

\subsubsection{\label{new_keyword}`this' keyword with `new'}
When using the `this' inside a traditional function and using the `new' operator while calling this function,
`this' gets bound to a new empty object which the function automatically returns.
The type of function that is used solely for this purpose is known as a `constructor':
\begin{verbatim}
    function Obj() {
        this.name = "foo";
        this.sayName = () => {
            console.log("My name is " + this.name);
        }
    }

    const obj = new Obj();
    obj.sayName(); // My name is foo
\end{verbatim}

\subsection{\label{defining_variables}Different ways of defining variables}
JavaScript variables can be defined in four ways as follows:

\subsubsection{Using const}
Using the const keyword is similar to using let but the variable can not be reassigned:
\begin{verbatim}
    function x() {
        for (let i = 0; i < 3; ++i) {
            const a = 'test';
            // a += i; is not possible
            console.log(a);
        }
        // console.log(a); is not possible here
    }
    x();
    // console.log(a); is not possible here
    // Expected output:
    // test
    // test
    // test
\end{verbatim}

\subsubsection{Using let}
Using let is similar to other programming languages:
it declares a variable which can only be seen in the block scope it was declared in.
\begin{verbatim}
    function x() {
        for (let i = 0; i < 3; ++i) {
            let a = 'test';
            a += i; // now possible
            console.log(a);
        }
        // console.log(a); is not possible here
    }
    x();
    // console.log(a); is not possible here
    // Expected output:
    // test0
    // test1
    // test2
\end{verbatim}

\subsubsection{Using var}
The var keyword declares a variable that is local to the function's scope,
but its declaration is executed before any code is executed.
This is known as `hoisting':
\begin{verbatim}
    function x() {
        for (let i = 0; i < 3; ++i) {
            var a = 'test';
            a += i; // now possible
            console.log(a);
        }
        console.log(a); // now possible
    }
    x();
    // console.log(a); is not possible here
    // Expected output:
    // test0
    // test1
    // test2
    // test2
\end{verbatim}

\subsubsection{Not using anything}
Not using any keyword before declaration assigns a property to the global window object:
\begin{verbatim}
    function x() {
        for (let i = 0; i < 3; ++i) {
            a = 'test';
            a += i; // now possible
            console.log(a);
        }
        console.log(a); // now possible
    }
    x();
    console.log(a); // now possible
    // Expected output:
    // test0
    // test1
    // test2
    // test2
    // test2
\end{verbatim}

\subsection{Equality}
There are different ways to compare two values in JavaScript:

\subsubsection{Strict equality}
Using `===' to compare two values does not do type conversion,
returns false when the values have different types,
returns false when NaN is compared to NaN,
and returns true when +0 is compared to -0.
\begin{verbatim}
    console.log('1' === 1); // false
    console.log(1 === 1); // true
    console.log(NaN === NaN); // false
    console.log(+0 === -0); // true
\end{verbatim}

\subsubsection{Loose equality}
Using `==' to compare two values does type conversion,
returns false when NaN is compared to NaN,
and return true when +0 is compared to -0.
\begin{verbatim}
    console.log('1' == 1); // true
    console.log(1 == 1); // true
    console.log(NaN == NaN); // false
    console.log(+0 == -0); // true
\end{verbatim}

\subsubsection{Object.is()}
Using Object.is(value1, value2) to compare two values does not do type conversions,
returns false when the values have different types,
returns true when NaN is compared to NaN,
and returns false when +0 is compared to -0.
\begin{verbatim}
    console.log(Object.is('1', 1)); // false
    console.log(Object.is(1, 1)); // true
    console.log(Object.is(NaN, NaN)); // true
    console.log(Object.is(+0, -0)); // false
\end{verbatim}

\subsection{Different ways of defining a function}
The example in \ref{arrow_functions} shows four different ways of defining a square function in JavaScript:
traditional function, arrow function, arrow function without braces in direct return,
and arrow function without parentheses around the unary parameter.
There is at least one more way of defining this function (anonymous function without arrow):
\begin{verbatim}
    const square = function(a) {
        return a * a;
    }
\end{verbatim}

\subsection{Different ways of initializing objects}
There are different ways to initialize an object in JavaScript.
They can be initialized using object literals (as shown in \ref{objects}),
constructor functions (as shown in \ref{new_keyword}),
a builder function:
\begin{verbatim}
    const buildPerson = (name, age) => ({
        name,
        age
    });

    console.log(buildPerson("foo", 13));
    // { name: "foo", age: 13 }
    console.log(buildPerson("bar", 23));
    // { name: "bar", age: 23 }
\end{verbatim}
or a class:
\begin{verbatim}
    class Person {
        constructor(name, age) {
            this.name = name;
            this.age = age;
        }
    }

    console.log(new Person("foo", 13));
    // { name: "foo", age: 13 }
    console.log(new Person("bar", 23));
    // { name: "bar", age: 23 }
\end{verbatim}

\subsection{Different ways of setting members in object literals}
The property name in object literal can be defined with or without quotations.
However, if it contains special characters, it must be defined with quotations.
If a property name is the same as the property value, the value can be omitted:
\begin{verbatim}
    const gender = "male";
    const obj = {
        name: "foo",
        "age": 14,
        "marital-status": "single",
        gender
    }
\end{verbatim}

\subsection{Different ways of accessing object members}
A value of a property of an object can be accessed using the dot notation or square brackets.
Accessing the `obj' object values in the previous section:
\begin{verbatim}
    console.log(obj.name); // foo
    console.log(obj["age"]); // 14
\end{verbatim}

\section{Syntax design}
The syntax design of a programming language impacts its readability.
If a programming language has meaningful identifier names and special words, the readability increases.
Due to JavaScript's fast-growing nature and strive for backward compatibility,
some syntax design choices were not meaningful while others were.
The following are some examples of syntax design characteristics in JavaScript:

\subsection{let, var, const have misleading meanings}
`let', `var', `const' have been discussed in \ref{defining_variables} from a simplicity perspective.
The following subsections, discuss them from a syntax design perspective:

\subsubsection{var}
`var' is used to hoist a variable;
that is, the declaration becomes local to the function scope and executes before any code is executed.
However, the keyword `var' is shorthand for `variable' which means a name associated with a value that can change.
The keyword does not hint at the fact that the variable is hoisted.

This is mainly because `var' was the only way to declare a JavaScript variable till 2015,
and hoisting was how JavaScript variables were declared.
With the introduction of new ways to declare variables such as `let' and `const',
the `var' keyword was not changed due to the desire maintain backward compatibility.

\subsubsection{let}
`let' is used to declare a block-scoped variable.
It is inspired from the mathematical keyword `let' as in: `let x be a number',
and from other programming languages.

The keyword `let' does not hint at the fact that the variable is block-scoped,
it was picked because JavaScript needed another keyword than
`var'\footnote{\raggedright \url{https://stackoverflow.com/questions/37916940/%
        why-was-the-name-let-chosen-for-block-scoped-variable-declarations-in-javascri}}.

\subsubsection{const}
`const' is used to declare a block-scoped constant.
However, unlike constants in other languages,
a constant in JavaScript means a name that can not be changed through reassignment.
Therefore, the properties of a `const' object and the items in a `const'
array\footnote{This follows that an array is an object whose properties are its items (\ref{objects}).}
can be changed:
\begin{verbatim}
    const x = 3;
    // x += 2; is not possible since it reassigns const x

    const arr = [1, 2, 3];
    arr[1] = 5;
    // possible - modifying the property of an object (array)
    arr.push(4);
    // possible - adding a property to an object (array)
    console.log(arr); // [1, 5, 3, 4]
\end{verbatim}

\subsection{Problematic automatic semicolon insertion}
In JavaScript, a semicolon is required after each statement; however,
if the programmer does not provide a semicolon, JavaScript will try to insert one automatically
It is usually recommended to always put a semicolon after each statement because
the automatic semicolon insertion (ASI) could be problematic; consider the following example:
\begin{verbatim}
    function square(a) {
        return
            a * a
    }

    console.log(square(2)); // undefined
\end{verbatim}
This happens because ASI is triggered after the `return' keyword,
and hence `a * a' becomes dead code and the function does not return anything.

Consider the following example where ASI is not triggered leading to possibly unexpected behavior:
\begin{verbatim}
    const say = (x) => {
        console.log(x)
        return x
    }

    // No ASI between Line X and Line Y
    const tell = say // Line X
    (1).toString() // Line Y
    // Program outputs "1"
\end{verbatim}
The program will output `1' although it might seem that we did not call the `say' function which has the `console.log'.
In addition to that, the variable `tell' now holds the value `1';
while a programmer might expect it to hold the function `say'.
This is because `Line X' and `Line Y' got translated to:
\begin{verbatim}
    const tell = say(1).toString();
\end{verbatim}

\subsection{Re-assigning variables that are exposed by JavaScript}
JavaScript exposes some variables such as the global `window' object
which represents the window in which a script is running,
and the `arguments' object which is exposed to all functions and represents the arguments passed to them.
These variables can be re-assigned in JavaScript which can decrease readability.

Consider the following example which reassigns the arguments object.
\begin{verbatim}
    function normalArguments(a) {
        console.log(arguments[0]);
        console.log(a);
    }

    function modifiedArguments(a) {
        arguments[0] = "fake";
        console.log(arguments[0]);
        console.log(a);
    }

    function reassignedArguments(a) {
        arguments = ["fake"];
        console.log(arguments[0])
        console.log(a);
    }

    normalArguments("foo");
    // foo
    // foo
    modifiedArguments("foo");
    // fake
    // fake
    reassignedArguments("foo");
    // fake
    // foo
\end{verbatim}

\section{Exception handling}
Exception handling is the ability of a program to intercept runtime errors, take corrective actions and
proceed\footnote{\raggedright
    \url{https://github.com/aminallam/aminallam.github.io/blob/master/pdf/concepts/Lecture01_Introduction.pdf}
}.

In JavaScript,
The code that is expected to raise an exception is surrounded with try..catch, try..finally or try..catch..finally.
JavaScript tries to execute the code in the `try' block.

If the code raises an exception, the control flow is moved to the catch block.
The code in the finally block executes regardless whether the code raised an exception or didn't
and whether there existed a catch block that caught this exception or not.

The `throw' statement can be used to throw a user-defined exception.
\begin{verbatim}
    class MyException {}

    const throwExceptions = (a) => {
        try {
            a.toUpperCase();
            if (a == "foo")
                throw new MyException();
            x();
        } catch (e) {
            // a.toUpperCase() fails
            if (e instanceof TypeError)
                console.log("Bad type");
            // "throw new MyException()" executes
            else if (e instanceof MyException)
                console.log("Bad value");
            // x(); throws a ReferenceError
            else {
                console.log("Unknown");
                throw e;
            }
        } finally {
            console.log("Cleaning up");
        }
    }

    throwExceptions(1);
    // Bad type
    // Cleaning up
    throwExceptions("foo");
    // Bad value
    // Cleaning up
    throwExceptions("bar");
    // Unknown
    // Cleaning up
    // ERROR (ReferenceError)
\end{verbatim}
Notice that due to JavaScript's dynamic typing,
we had to use if conditions to check for the exception type after catching it
and to re-throw it if it is an unexpected type.
This is different to how exceptions are caught in languages like C++ and Java
where the exception type is mentioned explicitly in the catch statement.

\end{document}
